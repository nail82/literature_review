\documentclass{article}
\begin{document}
Horn and Schunk proposed early techniques for the determination of optical flow in image sequences.  The problem domain they tackled in this work was simplified by assuming uniform illumination, smooth reflectance variation (for differentiability) and the notion that brigthness motion corresponds to object motion.  They point out that one component of optical flow velocity lies in direction of the brigthness gradient vector, however the other component (perpendicular to the gradient) cannot be determined without another constraint.  The additional constraint introduced by the authors was that of smoothness of flow and they developed an iterative computational method.


- low brightness gradient, optical flow at a point reduces to neighborhood mean velocity
- experimental work was with generated sequences to investigate convergence of the methd



\end{document}