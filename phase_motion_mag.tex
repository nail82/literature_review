\title{CS699 Literature Review}
\author{Ted Satcher}
\documentclass[letterpaper]{article}
\usepackage{cite}
\usepackage{setspace}
\date{\today}
\doublespacing

\begin{document}
\maketitle

\section{Phase Based Motion Magnification}
Wadhwa et al in~\cite{phase-based} introduce a new method for
amplifying small motions in video sequences.  The defining
characteristic of this technique is the use of temporal variations of
phase information for the detection of motion.  The authors make use
of steerable pyramids described by Simoncelli and Freeman
here~\cite{steer-pyr} and complex-valued Gabor filters for measuring
phase information.  The use of phase information provides several
benefits including reduced noise in the amplified sequence and the
potential for increased magnification factors.  The primary drawback
of the technique is performance.  Each filter orientation used in the
pyramid representation has both real and imaginary components, creating
a representation of the sequence with a number of intermediate images
per frame of the sequence.  The paper concludes with an extensive
experimental section exploring the qualitative performance and
limitations of the method.










\bibliographystyle{IEEEtran}
\bibliography{IEEEabrv,proposal}

\end{document}