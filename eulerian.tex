\title{CS699 Literature Review}
\author{Ted Satcher}
\documentclass[letterpaper]{article}
\usepackage{cite}
\usepackage{setspace}
\date{\today}
\doublespacing

\begin{document}
\maketitle

\section{Eulerian Video Magnification}
The paper by Wu et. al~\cite{eulerian} describes a technique for the
amplification of small motions in a video sequence.  The process
developed by the authors involves creating a multiscale representation
of each frame in the sequence and then bandpass filtering the temporal
intensity variations of groups of pixels in the representation. The
filtered representation is linearly amplified and the video is
reconstructed enabling the visualization of hidden motion and color
changes in the video.  This approach differs from earlier techniques,
for instance Liu~\cite{Liu-lagrange}, in that no attempt is made to
extract the motion field.  The benefits to this approach are
simplified computation and reassembly of the magnified sequence.  The
primary drawback is the introduction of artifacts to the reconstructed
sequence when too large a magnification factor is applied.


\bibliographystyle{IEEEtran}
\bibliography{IEEEabrv,proposal}

\end{document}