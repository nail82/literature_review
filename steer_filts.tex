\title{CS699 Literature Review}
\author{Ted Satcher}
\documentclass[letterpaper]{article}
\usepackage{cite}
\usepackage{setspace}
\date{\today}
\doublespacing

\begin{document}
\maketitle

\section{Steerable Filters}
Freeman and Adelson in~\cite{steer-filters} describe techniques for
steering functions and they begin with a definition of steerability.
From their definition, a steerable function is one that can be
represented as a linear combination of rotated versions of itself.
This is a useful property in digital filters since it allows one to
compute filter outputs for arbitrary rotations given a set of
so-called basis functions. The authors go on to present theorems that
identify steerable functions and provide bounds on the number of basis
functions required to enable steerability. A good cross section of
application examples is provided. These include a multiscale image
decomposition called a steerable pyramid, orientation, texture and
contour analysis, and a shape from shading example.  Finally, they
extend the idea of steerability to three dimensions.



\bibliographystyle{IEEEtran}
\bibliography{IEEEabrv,proposal}

\end{document}