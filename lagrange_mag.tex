\title{CS699 Literature Review}
\author{Ted Satcher}
\documentclass[letterpaper]{article}
\usepackage{cite}
\usepackage{setspace}
\date{\today}
\doublespacing
\begin{document}
\maketitle

\section{Liu Motion Magnification}
Liu, in ~\cite{Liu-lagrange}, describes a motion magnification process
based on optical flow computations.  This is a complex method that
makes multiple passes over the input sequence with user interaction
required at several points.  The primary goal of the process is to
group pixels into motion clusters which are used to segment the
frames.  User interaction is required to help guide the segmentation
and to identify clusters whose motion should be magnified.  The
clustering process has two iterations of feature identification and
flow vector computations, one for frame registration (to remove camera
motion) and one to identify the motion clusters.  Collecting pixels
into coherent motion clusters is a primary computational task and
machine learning is used to improve performance at occlusion
boundaries.  The actual magnification is a simple scaling factor
chosen by the user.  A significant drawback of this approach is
processing time.  The paper reports a run time of ten hours.

\bibliographystyle{IEEEtran}
\bibliography{IEEEabrv,proposal}


\end{document}