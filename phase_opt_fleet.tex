\title{CS699 Literature Review}
\author{Ted Satcher}
\documentclass[letterpaper]{article}
\usepackage{cite}
\date{\today}
\begin{document}
\maketitle
\section{Fleet and Jepson Component Velocity from Phase}
The paper by Fleet and Jepson~\cite{phase-opt-flow} deals with the
computation of component velocity in image sequences.  The authors
identify issues that should be addressed when designing a method for
measuring velocity.  These include accuracy, robustness, space-time
localization and the determination of multiple velocities in a
neighborhood.  Fleet and Jepson go on to explain how typical methods
using amplitude fail in one or more of these areas.  Section two
describes the filters used to analyze the sequence.  The filters are a
family of three dimensional Gabors whose attractiveness lies in their
minimization of uncertainty between the spatiotemporal domain and the
frequency domain.  Section three explores a one dimensional velocity
example to demonstrate the superiority that a phase representation has
over an amplitude representation.  Subsequent sections detail more
complex experiments using synthetic and natural image sequences.

\bibliographystyle{IEEEtran}
\bibliography{IEEEabrv,proposal}

\end{document}