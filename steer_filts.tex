\title{CS699 Literature Review}
\author{Ted Satcher}
\documentclass[letterpaper]{article}
\usepackage{cite}
\usepackage{setspace}
\date{\today}
\doublespacing

\begin{document}
\maketitle

\section{Steerable Filters}
Freeman and Adelson in~\cite{steer-filters} describe techniques for
steering functions and they begin with a definition of steerability.
From their definition, a steerable function is one that can be
represented as a linear combination of rotated versions of itself.
This is a useful property in digital filters since it allows one to
compute filter outputs for arbitrary rotations given a set of
so-called basis functions.  The authors go on to present
theorems that identify steerable functions and provide bounds on the
number of basis functions required to enable steerability.





Enumerated in the paper are application areas
Important to design separable filters
Important to have as few basis functions as possible to minimize computation.



\bibliographystyle{IEEEtran}
\bibliography{IEEEabrv,proposal}

\end{document}