\title{CS699 Literature Review}
\author{Ted Satcher}
\documentclass[letterpaper]{article}
\usepackage{cite}
\usepackage{setspace}
\date{\today}
\doublespacing

\begin{document}
\maketitle

\section{Lucas-Kanade 20 Years On: A Unifying Framework}
This paper by Baker and Matthews~\cite{lk-20-yrs} is the first in a
four part series examining image alignment algorithms based on the
Lucas-Kanade method from~\cite{lk-81}.  The algorithms in this class
are iterative in nature and rely on gradient descent methods to
minimize the difference between a template image and a warped input
image.  The goal of this paper is to formulate a categorization
framework and to anayze example algorithms from different regions of
the framework in terms of convergence rate, convergence iterations,
computational complexity and noise robustness.  The authors also
present their modification to Lucas-Kanade, demonstrate its
equivalence to Lucas-Kanade and prove the computational advantage of
their method.  Finally they examine several examples of gradient
descent methods in the context of their image alignment technique.




%   One axis of
% their framework defines the direction attribute of an algorithm.  A
% forward algorithm warps the input image back to the coordinate frame
% of the template and conversely, an inverse algorithm warps the
% template back to the input image. A second axis identifies
% algorithms as having either additive or compositional updates.
% Additive updates iteratively modify the parameters of a warp and
% compositional updates

\bibliographystyle{IEEEtran}
\bibliography{IEEEabrv,proposal}

\end{document}