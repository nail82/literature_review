\title{CS699 Literature Review}
\author{Ted Satcher}
\documentclass[letterpaper]{article}
\usepackage{cite}
\usepackage{setspace}
\date{\today}
\doublespacing

\begin{document}
\maketitle

\section{Shiftable Multiscale Transforms}
The multiscale framework described by Simoncelli, et al
in\cite{shift-multi} is meant to address an undesirable behavior in
wavelet and other critically sampled multiscale transformations.  A
critically sampled transformation is one in which the number of
coefficients in the transformation is equal to the number of samples
in the signal.  The behavior in question relates to coefficient power
in the subbands of a wavelet transform and exhibits itself when the
transformed signal is spatially shifted.  A simple spatial shift will
cause the coefficient power to radically jump around in the subbands.
The goal of the method in this paper is to formulate a small set of
basis functions from a signal that permit the interpolation of the
multiscale transform given any signal shift in spatial location,
orientation or scale.  The approach used by the authors is to
mathematically define shiftability and then set about to compute the
requirements and properties of a shiftable transformation.  In the
experimental section, the authors apply the technique to a synthetic
stereo matching problem and compare the qualitative performance of
their technique versus a wavelet approach.




\bibliographystyle{IEEEtran}
\bibliography{IEEEabrv,proposal}

\end{document}