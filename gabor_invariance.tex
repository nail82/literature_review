\title{CS699 Literature Review}
\author{Ted Satcher}
\documentclass[letterpaper]{article}
\usepackage{cite}
\usepackage{setspace}
\date{\today}
\doublespacing

\begin{document}
\maketitle

\section{Invariance Properties of Gabor Filter-based Features}
Kamarainen, Kyrki and Kalviainen in~\cite{k-paper} present an
excellent overview of Gabor filters and their usage in image feature
extraction settings.  They begin with a general expression of
one-dimensional Gabors and then extend that expression to two
dimensions, both in the spatial and frequency domains.  The authors
make several important observations with respect to the construction
and use of Gabor filters including an expression of minimum filter
size, filter parameter selection and image location where filter
response becomes unreliable.  The bulk of the paper is dedicated to
the invariance properties inherent in Gabor filters and techniques to
utilize these properties.  Gabors can be made invariant to
translation, orientation, scale and illumination.  They also
demonstrate in several examples the superior noise performance of
Gabors over other methods.




\bibliographystyle{IEEEtran}
\bibliography{IEEEabrv,proposal}

\end{document}